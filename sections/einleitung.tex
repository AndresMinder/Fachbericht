\section{Einleitung}
Pflanzen benötigen eine ihnen entsprechende Umwelt. Durch meteorologische Messdaten kann diese ermittelt und durch Agronome optimal bewirtschaftet werden. Ausserdem können bei genügend Messdaten aus einem Erfassungsnetz Wetterprognosen erstellt werden. Die Erhebung solcher Messdaten trägt somit erheblich zum wirtschaftlichen Erfolg in der Agronomie bei und kann bei einem geeignet grossen Erfassungsnetz Bewohner vor Unwetter warnen. In den ärmeren Teilen Afrikas und Südamerikas sind solche Systeme jedoch nicht verbreitet.\\[0.5cm]
Aus diesem Grund soll eine kostengünstige, erweiterbare und mobile Wetterstation gebaut werden, welche die örtlichen Agronomen unterstützt. Diese Wetterstation soll die Niederschlagsmenge, die Windstärke, die Lufttemperatur und die Sonnenstunden messen können. Ausserdem soll die Wetterstation mittels Photovoltaik unterstützt werden, und erhobene Daten via SMS abrufbar sein. Mit einem GPS-Modul soll der Standort der mobilen Wetterstation erfasst werden. Im Projekt 5 wird ein erster Prototyp erstellt, welcher die Datenermittlung mit der Sensorik und Datenspeicherung beinhaltet. Die Stromversorgung (Akku, Solarpanels), das GPS und das Senden der Daten über SMS werden im Projekt 6 implementiert.\\[0.5cm]
Die Lufttemperatur wird über den Temperatursensor XYZ gemessen. Mittels Kipplöffelprinzip wird die Niederschlagsmenge über den Sensor XYZ ermittelt, wobei die Funktionsweise des Kipplöffels zuerst mit einem Selbstbau überprüft wird. Die Windstärke wird über das Anemometer XYZ eruiert. Um die Sonnenstunden zu ermitteln werden zwei Varianten (zum einen via Ladestrom der Photovoltaikanlage, zum anderen mit einem Sensor XYZ) verglichen und die Leistungsärmere implementiert, was jedoch erst im Projekt 6 stattfindet. Es wird für das Prototyping ein ArduinoMega2560 verwendet, welcher die Sensoren über die vorhandenen Peripherieanschlüsse mit dem Mikrocontroller verbindet. Die gesammelten Daten werden auf einer microSD Karte abgespeichert, wobei mit Hilfe eines RTC (Real Time Clock) ein Zeitstempel hinzugefügt wird.\\[0.5cm]
Das Projekt 5 liefert die erwünschten Messdaten von Lufttemperatur, Windgeschwindigkeit und Niederschlagsmenge. Ausserdem werden die Daten wunschgemäss auf einer microSD Karte, mit zugehörigem Zeitstempel versehen, gespeichert.

%NOTE:
%XYZ = Platzhalter für Namen
