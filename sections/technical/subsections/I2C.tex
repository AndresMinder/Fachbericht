\subsection{I$^{2}$C}
\label{subsubsec:I2C}
Das \textbf{I$^{2}$C} (\textit{inter integrated circuit}) ist ein synchrones serielles Datenprotokoll (Datenbus) der aus zwei bidirektionalen Leitungen besteht. Dabei wird eine der Leitungen für den transfer des Taktsignals (\textbf{SCL}, \textit{serial clock line}) und die andere für den Datentransfer (\textbf{SDA}, \textit{serial data line}) verwendet. Dieser Bus wurde speziell dafür entwickelt, dass \textbf{ICs} (\textit{integrated circuits}) über kleine Distanzen gut miteinander kommunizieren können. Es können bis zu 128 \textbf{ICs} (7-Bit-Adressierung) über einen \textbf{I$^{2}$C}-Bus miteinander verbunden werden, wobei mindestens einer davon Master sein muss. Nur Master können aktiv Slaves ansprechen, da die Slaves nicht in der Lage sind selbstständig mit der Kommunikation zu beginnen. Um die Kommunikation mit einem Slave einzuleiten gibt der Master die eindeutige Adresse des \textbf{ICs} auf die Datenleitung und teilt zugleich mit, ob er Daten senden oder empfangen möchte. Die Daten werden im Anschluss direkt gesendet und bei Beendigung des Sendevorgangs wird der Bus wieder freigegeben. Bei mehreren Mastern gewährleistet ein Protokoll die sichere Kommunikation durch das verhindern von Kollisionen. Die einfache Ansteuerung ist ein grosser Vorteil des \textbf{I$^{2}$C}-Bus. Ausserdem können sowohl langsame als auch sehr schnelle \textbf{ICs} eingesetzt werden. Neuere Formen des \textbf{I$^{2}$C}-Bus ermöglichen eine 10-Bit-Adressierung, welche ebenso mit 7-Bit-Adressierungsschema kombiniert werden können, was die Skalierbarkeit des Bus nochmals erhöht. Ausserdem ist es mit neuen Bauelementen möglich, mehrere identische \textbf{ICs} mit derselben Adresse anzuschliessen, was die Skalierbarkeit des Bus nochmals steigert. \cite{mcI2C} \cite{rnI2C}\\