In diesem Kapitel wird die Validierung des Gesamtkonzepts beschrieben. Da das Gesamtkonzept aus mehreren separaten Teilstücken besteht, reicht es aus die Teilsysteme auch separat zu testen. Zum einen sind da die jeweiligen Sensoren (für Lufttemperatur, Windgeschwindigkeit, Niederschlagsmenge, Luftfeuchtigkeit, Luftdruck und Windrichtung), zum anderen ist da das Speichern der Messdaten auf der $\mu$SD-Karte mit Zeitstempel. 

\subsection{Validierung der Sensorik}
Die verschiedenen Sensoren besitzen Abweichungen, welche aus den entsprechenden Datenblättern entnommen sind und in den jeweiligen Kapitel thematisiert wurden. Dennoch müssten die Messwerte zusätzlich verifiziert werden. Um diese zu verifizieren wäre es jedoch notwendig, sehr genaue Referenzwerte zu haben über die gesamten Messbereiche. Da zu diesem Zeitpunkt keine Messvorrichtungen zu Verfügung stehen welche über entsprechenden Genauigkeiten besitzen, um Aussagekräftige Verifizierungen zu ermöglichen, konnte die Sensorik nicht validiert werden.

\subsection{Validierung der Firmware}
Die Validierung der Firmware soll die Funktionsweise des Programms bestätigen. Das Gesamtprogramm wurde getestet, indem durch die Sensorik Messdaten generiert, welche dann mit Zeitstempel auf die $\mu$SD-Karte gespeichert und von dort aus über die USB-Schnittstelle ausgegeben wurden. Durch die Abbildung \ref{fig:datenausgabe}, aus dem Kapitel \ref{subsec:proto}, ist ersichtlich, dass die Funktion des Programms gewährleistet ist. Das Programm selbst besitzt dennoch Verbesserungspotenzial. Bei näherer Betrachtungsweise des UML-Diagramms (Abbildung \ref{fig:uml_diagramm}) kann eine Redundanz festgestellt werden, welche herausgenommen werden kann.