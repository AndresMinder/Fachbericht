\section{Schluss}

Wetterstationen liefern wichtige Klima- und Wetterdaten für örtliche Agronome. In subtropischen Gebieten wie Teile Afrikas und Südamerikas besitzen die Agronome diese Daten nicht, weshalb eine möglichst günstige, mobile Wetterstation entwickelt werden soll um diese zu Unterstützen. In einem ersten Projekt soll die Sensorik und die Datenspeicherung dazu entwickelt werden.\\[0.5cm]
Die Lufttemperatur, Luftdruck und Luftfeuchtigkeit wird über ein BME280 von Bosch gemessen. Die Windgeschwindigkeit über ein Anemometer von Froggit. Für die Windrichtung wird ein Windrichtungsgeber von Misol ermittelt. Über ein Ombrometer von Misol wird die Niederschlagsmenge eruiert. Die gesammelten Messdaten werden mit einem Zeitstempel, generiert von einem integrierten Präzisions-RTC (DS3231), versehen und auf eine $\mu$SD-Karte gespeichert. Kern der mobilen Wetterstation bildet eine MCU (ArduinoMega2560), welche über eine USB-Schnittstelle ansteuerbar ist.\\[0.5cm]
Die Lufttemperatur wird im Bereich [0;60]$^\circ$C mit einer Abweichung von $\pm$ 1$^\circ$C ermittelt und in einem Bereich von [-20;0]$^\circ$C mit einer Abweichung von $\pm$ 1.25$^\circ$C. Die Luftfeuchtigkeit wird mit mit einer absoluten Genauigkeitstoleranz von $\pm$3 \% gemessen. Die Luftdruckmessung erfolgt mit einer absoluten Genauigkeit von $\pm$1.7 hPa in einem Temperaturbereich von [-20;0]$^\circ$C und mit $\pm$1 hPa in einem Temperaturbereich von [0;65]$^\circ$C. Die Windrichtung kann in Nord, Süd, West, Ost, Nordost, Südost, Südwest und Nordwest eingeteilt werden. Das Ombrometer misst auf einer Trichterfläche von 6.325e-3 m$^2$ ungefähr 2 ml pro Kippbewegung. Das Anemometer misst eine nicht verifizierte Windstärke. Die Datenspeicherung mit Zeitstempel via MCU funktioniert einwandfrei.\\[0.5cm]
In einem weiteren Projekt wird die Energieversorgung (Akku mit Solarunterstützung), die Kommunikation (Datenabfrage via SMS, GPS), sowie die Ermittlung der Sonnenstunden entwickelt. Wir empfehlen, dass die entwickelte Sensorik mit Hilfe von präzisen Messvorrichtungen verifiziert wird. Vor allem um die Windgeschwindigkeit, welche das Anemometer misst, verifizieren zu können. Ausserdem soll die Trichterfläche des Ombrometers vergrössert werden, damit eine genauere Angabe auf eine Referenzfläche von 1 m$^2$ erfolgen kann. Die Richtungsangabe des Windrichtungsgebers muss ebenso verbessert werden, damit die 8 genannten Windrichtungen besser erkannt werden.\\