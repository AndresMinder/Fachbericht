\section{Sensoren}
<<<<<<< HEAD

\input{sections/technical/subsections/ombrometer}
=======
\label{chap:Sensoren}
\subsection{Messen der Lufttemperatur}
\subsection{Ermittlung der Niederschlagsmenge}
Dieses Unterkapitel befasst sich mit der Realisierung der Niederschlagsmessung. Diese soll nach einem Kipplöffelprinzip funktionieren und gemäss definierten Zielen eine Genauigkeit von $\pm$100 ml/$m^2$ aufweisen. Ausserdem soll als alternative zusätzlich ein Messbecher an der Wetterstation installiert werden, damit der Bauer die Niederschlagsmenge anhand einer Skala ablesen kann. In einem ersten Schritt soll das Kipplöffelprinzip näher erläutert und mit einem Selbstbau die Funktionsweise getestet werden. Anschliessend soll ein gekaufter Sensor die Wetterstation erweitern und die Implementation in der Firmware thematisiert werden. Zu guter Letzt soll die Validierung des Teilsystems folgen.
\subsubsection*{Das Kipplöffelprinzip}
Das Prinzip des Kipplöffels wird in Abbildung \ref{fig:Kipp} graphisch dargestellt.
>>>>>>> master

\newpage
\subsection{Anemometer}
{\begin{minipage}[b][650pt][t]{0.55\textwidth}
Für die Windgeschwindigkeitsmessung wurde ein Ersatz Anemometer von Froggit genommen (Abb. \ref{fig:anemometer}). Das Anschlusskabel hat einen vier poligen RJ-11 Stecker, dessen Signal über eine Buchse zum MCU geführt wird. Das Anemometer selbst hat allerdings nur zwei Anschlüsse, die Speisung (rot) und das durch einen mit einem Dauermagneten schließbaren Reedkontakt modulierte pulsförmige Ausgangssignal (grün, Abb. \ref{fig:rj11stecker}). In der Abb. \ref{fig:beschaltungAnemometer} ist ersichtlich, dass das Ausgangssignal über R1 abfällt und C1 als Spannungsstabilisierung dient. Das daraus resultierende Signal ist in der Abb. \ref{fig:rechteckpuls_anemometer} aufgezeigt. Die Windgeschwindigkeit ist nun aus der Anzahl Rechteckpulsen direkt interpretierbar:\\

Wenn über einen Zeitraum $T$ die Anzahl Pulse $A$ gemessen werden, dann kann auf die Winkelgeschwindigkeit $\omega$ nach 
\begin{equation}
\centering
\omega=\frac{A}{T}\qquad[s^{-1}]
\end{equation}
geschlossen werden. Da allerdings verschiedene Faktoren wie das Trägheitsmoment des Schalenkreuzes, Reibungsverluste bei der Drehbewegung, Verfälschung bei wechselnder Windrichtung usw. zusätzlich auf das Anemometer wirken, wird es sehr komplex die Windgeschwindigkeit exakt zu berechnen. Deshalb wird nur ein Näherungswert ermittelt und mit einem Skalierungsfaktor $SF$ korrigiert. Somit ergibt sich für die Windgeschwindigkeit $v_{Wind}$ mit Radius $r$ des Schalenkreuzes
\begin{equation}
\centering
v_{Wind} = \frac{A*r*SF}{T}\qquad[m/s].
\label{equ:berechnungWindgeschwindigkeit}
\end{equation}
Der Skalierungsfaktor $SF$ wird mittels Referenzmessungen der Windgeschwindigkeit eines digitalen Anemometers eruiert. \\
\end{minipage}}
{\begin{minipage}[b][650pt][t]{0.44\textwidth}
\centering
\includegraphics[width=0.9\textwidth]{graphics/Anemometer/anemometer.png}
\captionof{figure}{Anemometer \cite{AmazonAnemometer}}
\label{fig:anemometer}
\vspace{20pt}
\includegraphics[width=0.9\textwidth]{graphics/Anemometer/rj_11_anschlussstecker.png}
\captionof{figure}{RJ-11 Stecker}
\label{fig:rj11stecker}
\vspace{20pt}
\includegraphics[width=0.9\textwidth]{graphics/Anemometer/schaltung_anemometer.png}
\captionof{figure}{Beschaltung des Ausgangs des Anemometers.}
\label{fig:beschaltungAnemometer}
\vspace{20pt}
\includegraphics[width = 0.9\textwidth]{graphics/Anemometer/oszilloskop_anenometer_puls.png}
\captionof{figure}{Ausgangssignal $V_{out}$}
\label{fig:rechteckpuls_anemometer}
\end{minipage}}
\newpage

\subsubsection{Implementation in die Firmware}
Die Implementation wurde recht simpel gehalten. Der gesamte implementierte Code für das Anemometer ist im Headerfile ''Anemometer.h'' extern deklariert und im File Anemometer.cpp initialisiert. Das Signal $V_{out}$ ist mit einen digital Pin des Atmega 2560 (Pinnummer 2 des Arduino Mega Boards) verbunden. Über einen Zeitraum von $5000ms$, auf die steigende Flanke getriggert, wird die Anzahl von Pulsen mittels Interrupt\footnote{es handelt sich hierbei um \textit{external Interrupts}.} gezählt. Dabei wird zuerst der Interrupt auf der Pinnummer 2 aktiviert, mit einem Delay von $5000ms$ gewartet, wobei bei jedem ausgelösten Interrupt die Funktion \textcolor{orange}{countWind}() ausgeführt und somit bei jeder steigenden Flanke um eins inkrementiert wird. Zum Schluss folgt die Deaktivierung des Interrupts und die Berechnung der Windgeschwindigkeit nach der Gleichung \ref{equ:berechnungWindgeschwindigkeit}.\\

\subsubsection{Validierung}
Über eine einigermaßen konstanten Windgeschwindigkeit eines Heizlüfters/Ventilators (mit verschiedenen Stärkestufen) wurden Messpunkte des Anemometers (Abb. \ref{fig:anemometer}), sowie auch des digitalen Anemometers (Abb. \ref{fig:digitalesAnemometer}) erfasst. In der Abb. \ref{fig:messpunkteAnemometer_Vergleich} sind diese Messwerte graphisch dargestellt.\\

{\begin{minipage}[b][240pt][t]{0.65\textwidth}
\centering
\includegraphics[width=\textwidth]{graphics/placeholder.png}
\captionof{figure}{Graph der Messwerte}
\label{fig:messpunkteAnemometer_Vergleich}
\end{minipage}}
{\begin{minipage}[b][240pt][t]{0.34\textwidth}
\centering
\includegraphics[width=0.9\textwidth]{graphics/Anemometer/messgeraet_anemometer.png}
\captionof{figure}{Digitales Anemometer (Benetech GM8908) mit einer Auflösung von $0.1m/s$ und einer Unsicherheit von $\pm5\%$ \cite{digitalesAnemometerBenetech}}
\label{fig:digitalesAnemometer}
\end{minipage}}
\todo[inline]{Hier muss noch die Interpretation der gemessenen Werte geschrieben werden.}
\newpage

\newpage

\subsection{Windrichtungsgeber}
{\begin{minipage}[b][10cm][t]{0.55\textwidth}
Um die Windrichtung angeben zu können, wurde ein Windrichtungsgeber, wie in Abb. \ref{fig:windrichtungsgeber} gezeigt von MISOL verwendet. Er ist genau wie das Ombrometer und das Anemometer mit Reedkontakten realisiert worden (siehe Abb. \ref{fig:interne_schaltung}). Dafür sind acht Reedkontakte im Kreis angeordnet und jeder hat einen in Serie geschalteten Widerstand von unterschiedlichen Dimensionen. Der Dauermagnet kann, je nach Drehwinkel bis zu zwei Reedkontakte gleichzeitig schließen. Dies erlaubt sechzehn verschiedene Winkelpositionen und somit eine Auflösung von 22.5$^{o}$. Mit einem externen Widerstand $R=10k\Omega$ (siehe Abb. \ref{fig:aussere_beschaltung}) wird eine Spannung generiert, welche dann mit dem ADC des Microcontrollers gelesen werden kann. Der Windrichtungsgeber wird mit einer Speisespannung von $V_{+}=5V$ betrieben. Der Windrichtungsgeber hat einen vierpoligen RJ-11 Anschluss. Zudem hat er auf der unteren Seite noch eine RJ-11 Buchse, bei der das Anemometer direkt angeschlossen werden kann. Wie diese Anschlüsse gemapped sind, wird in der Abb. \ref{fig:interne_schaltung} gezeigt. \\
\end{minipage}}
{\begin{minipage}[b][10cm][t]{0.44\textwidth}
\centering
\includegraphics[width=0.9\textwidth]{graphics/windrichtungsgeber/interne_schaltung.PNG}
\captionof{figure}{Interne Schaltung \cite{ADSkeineAngabe}}
\label{fig:interne_schaltung}
\end{minipage}}

\begin{table}[h]
\centering
\caption{Technische Werte \cite{ADSkeineAngabe}}
\begin{tabular}{|c|c|c|c|}
\hline 
Richtung [$^{o}$] & Himmelsrichtung & Widerstand [$\Omega$] & Ausgangsspannung [V] \\ 
\hline 
0 & N & 33k & 3.84 \\ 
\hline 
22.5 &  & 6.57k & 1.98 \\ 
\hline 
45 & NE & 8.2k & 2.25 \\ 
\hline 
67.5 &  & 891 & 0.41 \\ 
\hline 
90 & E & 1k & 0.45 \\ 
\hline 
112.5 &  & 688 & 0.32 \\ 
\hline 
135 & SE & 2.2k & 0.90 \\ 
\hline 
157.5 &  & 1.41k & 0.62 \\ 
\hline 
180 & S & 3.9k & 1.40 \\ 
\hline 
202.5 &  & 3.14k & 1.19 \\ 
\hline 
225 & SW & 16k & 3.08 \\ 
\hline 
247.5 &  & 14.12k & 2.93 \\ 
\hline 
270 & W & 120k & 4.62 \\ 
\hline 
292.5 &  & 42.12k & 4.04 \\ 
\hline 
315 & NW & 64.9k & 4.33 \\ 
\hline 
337.5 &  & 21.88k & 3.43 \\ 
\hline 
\end{tabular} 
\label{tab:technische_werte}
\end{table}

In der Tabelle \ref{tab:technische_werte} sind die Widerstandswerte der in Abb. \ref{fig:interne_schaltung} gezeigten Widerständen und die Werte der Ausgangsspannung bei variabler Winkelposition aufgelistet. Da nun abhängig von der Winkelposition unterschiedliche Widerstände parallel geschalten werden, resultiert am Ausgang eine vom Winkel abhängige Ausgangsspannung.\\

\newpage


{\begin{minipage}[b][6cm][t]{0.49\textwidth}
\centering
\includegraphics[width=0.5\textwidth]{graphics/windrichtungsgeber/aeussere_beschaltung.PNG} 
\captionof{figure}{Spannungsteiler mit $R=10k\Omega$ \cite{ADSkeineAngabe}}
\label{fig:aussere_beschaltung}
\end{minipage}}
{\begin{minipage}[b][6cm][t]{0.49\textwidth}
\centering
\includegraphics[width=0.9\textwidth]{graphics/windrichtungsgeber/windrichtungsgeber.PNG}
\captionof{figure}{Windrichtungsgeber von MISOL \cite{windrichtungsgeber}}
\label{fig:windrichtungsgeber}
\end{minipage}}

\subsubsection*{Implementation in die Firmware}
Der ADC hat des ATMega2560 hat eine 10 Bit Auflösung und wird mit 5 Volt betrieben. Um das Signal am ADC lesen zu können, wird die Funktion \textit{analogRead()} aufgerufen. Als Rückgabewert wird ein binärer Wert als Dezimalzahl vom Datentyp \textit{float} erhalten. Um diesen Wert in die eigentlich am ADC anliegende Spannung umzurechnen wird die Gleichung \ref{equ:ADC1} umgeformt.
\begin{equation}
\centering
\dfrac{Auflösung ADC}{Betriebsspannung} = \dfrac{analogRead()}{Ausgangsspannung}
\label{equ:ADC1}
\end{equation}
Daraus resultiert:
\begin{equation}
\centering
Ausgangsspannung = \dfrac{analogRead()*Betriebsspannung}{Auflösung ADC}
\label{equ:ADC2}
\end{equation}
Werden nun die Werte in die Gleichung \ref{equ:ADC2} eingefügt, ergibt sich für die Ausgangsspannung $V_{out}$:
\begin{equation}
\centering
V_{out}(analogRead()) = \dfrac{analogRead()*5V}{10}
\label{equ:adc_vout}
\end{equation}
Diese von \textit{analogRead()} abhängige Ausgangsspannung $V_{out}$ wird dann in \textit{if else} Anweisungen zu der Himmelsrichtung zugewiesen und als String abgespeichert.

%\input{sections/technical/subsections/sonnenstunden}

\newpage

<<<<<<< HEAD
\subsection{BME280}

{\begin{minipage}[b][6cm][t]{0.55\textwidth}
Der \textit{BME280} ist ein low powered digitaler Feuchtigkeits-, Luftdruck- und Temperatursensor von Bosch. Er ist in einem 2.5mm x 2.5mm x 0.93mm metal lid LGA Gehäuse verpackt und kann über die Interfaces I$^{2}$C und SPI kommunizieren. Durch seinen niedrigen Stromverbrauch, große operating range der drei Messgrößen und schnellen Ansprechzeit von etwa 1s eignet er sich für die solarbetriebene mobile Wetterstation besonders. \cite{Bosch2019}\\
\end{minipage}}
{\begin{minipage}[b][6cm][t]{0.44\textwidth}
\centering
\includegraphics[width=0.9\textwidth]{graphics/bme280/bme280.PNG}
\captionof{figure}{BME280 \cite{Bosch2019}}
\label{fig:bme280}
%\vspace*{0.5cm}
%\centering
%\includegraphics[width=0.9\textwidth]{graphics/bme280/bme280_pinout.PNG}
%\captionof{figure}{Pinout \cite{Bosch2019}}
%\label{fig:bme280_pinout}
\end{minipage}}

\begin{table}[h]
  \centering
  \caption{Elektrische Spezifikationen \cite{Bosch2019}}
    \begin{tabular}{lllll}
    \toprule
    \textbf{Parameter} & \textbf{Min.} & \textbf{Typ.} & \textbf{Max.} & \textbf{Einheit} \\
    \midrule
    Versorgungsspannung & 1.71  & 1.8   & 3.6   & V \\
    Stromverbrauch (sleep mode) &       & 0.1   & 0.3   & $\mu$A \\
    Stromverbrauch inaktiv (normal mode) &       & 0.2   & 0.5   & $\mu$A \\
    Stromverbrauch Feuchtigkeitsmessung &       & 340   &       & $\mu$A \\
    Stromverbrauch Luftdruckmessung &       & 714   &       & $\mu$A \\
    Stromverbrauch Temperaturmessung &       & 350   &       & $\mu$A \\
    \bottomrule
    \end{tabular}%
  \label{tab:elektrische_Spezifikationen}%
\end{table}%

Bei einer Messfrequenz von 1Hz für die drei Messgrößen verbraucht der BME280 somit laut Datenblatt nur \textbf{3.6$\mu$A}. \cite[S. 2]{Bosch2019}

\subsubsection*{Feuchtigkeitsmessung}
In der Tabelle \ref{tab:spez_feuchtigkeit} sind die wichtigsten Parameter zur Feuchtigkeitsmessung aufgelistet. Zu vermerken ist, dass die digitalen Werte des BME280 zur Feuchtigkeitsmessung relativ sind und deshalb prozentual angegeben werden. \\
\begin{table}[htbp]
  \centering
  \caption{Sezifikationen der Feuchtigkeitsmessung \cite{Bosch2019}}
    \begin{tabular}{lllll}
    \toprule
     \textbf{Parameter} & \textbf{Min.} & \textbf{Typ.} & \textbf{Max.} & \textbf{Einheit} \\
    \midrule
    Operating range & -40   & 25    & 85    & $^{o}$C \\
          & 0     &       & 100   & \% \\
    Absolute Genauigkeitstoleranz &       & $\pm$3 &       & \% \\
    Hysterese &       & $\pm$1 &       & \% \\
    Auflösung &       & 0.008 &       & \% \\
    Langzeitstabilität &       & 0.5   &       & \% pro Jahr \\
    \bottomrule
    \end{tabular}%
  \label{tab:spez_feuchtigkeit}%
\end{table}%

\newpage

\subsubsection*{Luftdruckmessung}
Die Genauigkeit der Luftdruckmessung ist an einen Temperaturbereich gebunden. Bei niedrigeren Temperaturen (<0$^{o}$C) weist der Sensor eine höhere Unsicherheit auf als bei Temperaturen von 0 bis 65 $^{o}$C (siehe Tabelle \ref{tab:spez_druck}). \\
\begin{table}[htbp]
  \centering
  \caption{Sezifikationen der Luftdruckmessung \cite{Bosch2019}}
    \begin{tabular}{llllll}
    \toprule
    \textbf{Parameter} & \multicolumn{1}{l}{\textbf{Temperaturbereich}} & \multicolumn{1}{l}{\textbf{Min.}} & \textbf{Typ. } & \multicolumn{1}{l}{\textbf{Max.}} & \textbf{Einheit} \\
    \midrule
    Operating range &       & \multicolumn{1}{l}{-40} & 25    & \multicolumn{1}{l}{85} & $^{o}$C \\
          &       & 300   &       & 1100  & hPa \\
    Absolute Genauigkeit & \multicolumn{1}{c}{-20 bis 0 $^{o}$C} &       & $\pm$1.7 &       & hPa \\
          & \multicolumn{1}{c}{0 bis 65 $^{o}$C} &       & $\pm$1  &       & hPa \\
    Auflösung &       &       & \multicolumn{1}{r}{0.18} &       & hPa \\
    Langzeitstabilität &       &       & $\pm$1  &       & hPa pro Jahr \\
    \bottomrule
    \end{tabular}%
  \label{tab:spez_druck}%
\end{table}%

\subsubsection*{Temperaturmessung}
Die Wetterstation wird hauptsächlich in einem Temperaturbereich von 0 bis 65 $^{o}$C betrieben, wodurch vom Sensor eine Unsicherheit von max. $\pm$1 $^{o}$C erreicht werden kann. \\
\begin{table}[htbp]
  \centering
  \caption{Sezifikationen der Temperaturmessung \cite{Bosch2019}}
    \begin{tabular}{llllll}
    \toprule
    \textbf{Parameter} & \multicolumn{1}{l}{\textbf{Temperaturbereich}} & \multicolumn{1}{l}{\textbf{Min.}} & \textbf{Typ. } & \multicolumn{1}{l}{\textbf{Max.}} & \textbf{Einheit} \\
    \midrule
    Operating range &       & \multicolumn{1}{l}{-40} & 25    & \multicolumn{1}{l}{85} & $^{o}$C \\
    Absolute Genauigkeit & \multicolumn{1}{c}{25 C} &       & $\pm$0.5 &       & $^{o}$C \\
          & \multicolumn{1}{c}{0 bis 65 C} &       & $\pm$1  &       & $^{o}$C \\
          & \multicolumn{1}{c}{-20 bis 0 C} &       & $\pm$1.25 &       & $^{o}$C \\
          & \multicolumn{1}{c}{-40 bis -20 C} &       & $\pm$1.5 &       & $^{o}$C \\
    Auflösung  &       &       & 0.01  &       & $^{o}$C \\
    \bottomrule
    \end{tabular}%
  \label{tab:spez_temp}%
\end{table}%

\subsubsection*{Implementation in die Firmware}
Um den BME280 vom Microcontroller aus ansteuern zu können, wurden zwei bereits existierende Librarys von Adafruit verwendet:
\begin{itemize}
\item Adafruit BME280 Library
\item Adafruit Unified Sensors
\end{itemize}
Anschließend konnten die Headerfiles <Adafruit\_Sensor.h> und <Adafruit\_BME280> inkludiert werden und der Sensor über das I$^{2}$C Interface mit den folgenden Funktionen abgefragt werden:\\
\begin{itemize}
\item \textcolor{blue}{float} \textcolor{orange}{readTemperature}()
\item \textcolor{blue}{float} \textcolor{orange}{readHumidity}()
\item \textcolor{blue}{float} \textcolor{orange}{readPressure}()
\end{itemize}
=======
\paragraph{Nachteile des Selbstgebauten Niederschlagsmengensensor}
Der selbstgebaute Niederschlagssensor beweist, dass das Prinzip des Kipplöffels funktioniert. Dennoch weist der Selbstbau Mängel auf. Der verwendete Permanentmagnet muss geklebt werden, weshalb dessen Magnetfeld massiv an stärke verliert und die Schaltung deshalb äusserst nahe angebracht werden muss. Ausserdem kam es, dadurch dass keine Werkstatt zugänglich war, zu Improvisation bei nahezu allen Fertigungsschritten, was zu unkalkulierbaren Abweichungen führt. Als Beispiel sei das Spiel der drehbaren Lagerung des Kipplöffels auf dem Gerüst angeführt, was jegliche Justierungsversuche der Niederschlagsmenge beeinflusst. Aus den genannten Gründen wird ein vorgefertigter Sensor mit Kipplöffelprinzip verwendet.
>>>>>>> master


