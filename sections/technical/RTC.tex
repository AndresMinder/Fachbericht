\section{RTC}
\label{chap:RTC}
Um die gewonnenen Messdaten mit einem Zeitstempel zu versehen, wird eine \textbf{RTC} (\textit{real time clock}) benötigt. Um eine möglichst kleine Abweichung zu haben, wird eine hohe Präzision vorausgesetzt. Ausserdem soll die \textbf{RTC} über das in Kapitel \ref{subsubsec:I2C} erwähnte \textbf{I$^2$C}-Bus angesteuert werden. Aus diesen Gründen wurde der \textbf{DS3231} implementiert, da dieser als Präzisions-\textbf{I$^2$C}-\textbf{RTC} die Anforderungen erfüllt. Die hohe Präzision des \textbf{DS3231} wird mit einem internen Temperatursensor erreicht, welcher Temperaturbedingte Abweichungen des Oszillators kompensiert. Zu sehen ist der \textbf{DS3231} mit seinen Anschlüssen in Abbildung \ref{fig:DS3231}.

\begin{figure}[h]
\centering
\includegraphics[width=0.4\linewidth]{graphics/DS3231.png}
\caption{\textbf{DS3231} mit seinen Anschlüssen \cite{DS3231DS}.}
\label{fig:DS3231}
\end{figure}

Die Anschlüsse des \textbf{DS3231} sind \textbf{VCC}, \textbf{GND}, \textbf{SCL}, \textbf{SDA}, \textbf{BAT}, \textbf{32K}, \textbf{SQW} und \textbf{RST}. \textbf{VCC} und \textbf{GND} werden für die Speisung benötigt. \textbf{SCL} und \textbf{SDA} sind Anschlüsse für den \textbf{I$^2$C}-Bus. \textbf{BAT} ist der positive Anschluss der Knopfbatterie, worüber deren Zustand kontrolliert werden, etwas anderes gespiesen oder eine andere Battery als Backup angeschlossen werden kann. \textbf{32K} ist ein Anschlusspin um den Output des 32kHz-Oszillators des \textbf{RTC} abzugreifen, was jedoch nicht verwendet wird. \textbf{SQW} ist ein zusätzlicher Output- oder Interrupt-Pin, welcher jedoch auch nicht verwendet wird. \textbf{RST} wird verwendet, um ein externes Element zu reseten oder als Indikator wenn die Hauptspeisung unterbrochen wird. \cite{DS3231DS}\\

\newpage
Die relevanten Spezifikationen des Chips sind in Tabelle \ref{tab:DS3231} aufgelistet.\\

\begin{table}[h]
\begin{tabular}{llllll}
\hline 
\textbf{Parameter} & \textbf{Min.} & \textbf{Typ.} & \textbf{Max.} & \textbf{Einheit} & \textbf{Condition} \\ 
\hline 
VCC & 2.3 & 3.3 & 5.5 & V &  \\ 
VBAT & 2.3 & 3 & 5.5 & V &  \\ 
Active Supply Current &  &  & 200 & $\mu$A & 3.63 V \\ 
 &  &  & 300 & $\mu$A & 5.5 V \\ 
Standby Supply Current &  & & 110 & $\mu$A & 3.63 V \\ 
 &  &  & 170 & $\mu$A & 5.5 V \\ 
Crystal Aging &  & $\pm$ 1 &  & ppm & First Year \\ 
 &  & $\pm$ 5 &  & ppm & 0-10 Years \\ 
Active Battery Current &  &  & 70 & $\mu$A & 3.63 V \\ 
 &  &  & 150 & $\mu$A & 5.5 V \\ 
Timekeeping Battery Current &  & 0.84 & 3 & $\mu$A & 3.63 V \\ 
 &  & 1 & 3.5 & $\mu$A & 5.5 V \\ 
\hline 
\end{tabular} 
\caption{Spezifikationen des \textbf{DS3231} \cite{DS3231DS}.}
\label{tab:DS3231}
\end{table}

Tabelle \ref{tab:DS3231} zeigt die für das Projekt relevanten Spezifikationen des \textbf{DS3231}. Wichtig ist, dass die Alterung des Oszillators zu einem Fehler führt, dieser jedoch im ppm-Bereich liegt und somit erst über viele Jahre hinweg bemerkbar wird.

\subsection{Implementation in die Firmware}
Für die Implementation wurde die bereits existierende Library \textbf{RTClib} von Adafruit in die Firmware integriert. Anschließend konnte das Headerfile <Adafruit\_RTClib.h> inkludiert werden und mit der Funktion \textit{TimeStamp getTimeStamp()} der aktuelle Zeitstempel abgerufen werden. \\
Beim flashen des Programms wird die RTC mit der Uhr des angeschlossenen Computers synchronisiert, weshalb es durch die Übertragungsverzögerung zu einer Abweichung kommt, welche der Dauer des flashens entspricht.